\documentclass{swputhesis}
\usepackage{hyperref}
\usepackage[nameinlink]{cleveref}  % after hyperref package

\addbibresource{thesis.bib}

% 图书分类号:在 https://www.clcindex.com/ 查询
\clc{TE319}
\ctitle{论文题目论文题目论文题目论文题目论文题目论文题目论文题目论文题目论文题目}
\etitle{TITLE TITLE TITLE TITLE TITLE TITLE}
\cmajor{学科专业}
\emajor{major}
\research{研究方向}
\cauthor{作者姓名}
\eauthor{author name}
\id{200000000001}
\csupervisor{导师姓名}
\esupervisor{supervisor name}
\cdate{2021年6月}
\edate{June, 2021}
\ckeywords{西南石油大学,博士论文,模板}
\ekeywords{swpu, phd thesis, template}

\begin{document}

% 封面
\makecover

% 摘要、目录页
\frontmatter

\begin{cabstract}
	摘要题头应居中,小二号黑体 (与章标题格式相同)。
	摘要要求为写实性的叙述,阐明研究意义、理论方法、开展的工作、取得成果和认识,最好给出定量性的结论。
	英文摘要应与中文内容一致,表述得体。

	摘要文字后空一行,顶格 (即不缩进) 写出关键词。``关键词:'' 用小四号、黑体,
	而关键词内容用小四号、宋体,内容 3 $\sim$ 5 个,关键词之间用  ``,'' 隔开。
\end{cabstract}

\begin{eabstract}
	The abstract is an important component of your thesis. Presented at the 
	beginning of the thesis, it is likely the first substantive description of 
	your work read by an external examiner. You should view it as an opportunity 
	to set accurate expectations.

	The abstract is a summary of the whole thesis. It presents all the major 
	elements of your work in a highly condensed form.
	An abstract often functions, together with the thesis title,
	as a stand-alone text.

	``Key words'' 没给出格式规定,几个 Word 模板都有不同,有采用宋体,有采用黑体的,
		这里暂时采用了宋体加粗。
\end{eabstract}


% 生成目录
\tableofcontents

% 正文
\mainmatter

% ==============================================================================
\chapter{论文格式要求}

% ------------------------------------------------------------------------------
\section{正文字体与段落}

\begin{enumerate}
	\item 正文字体: 小四号;
	\item 中文:宋体;
	\item 西文: Times New Roman;
	\item 行距 1.25 倍;
	\item 段前段后 0 行;
	\item 全文首行缩进 2 字符;
	\item 标点符号用全角 (学校未规定)。
\end{enumerate}

\subsection{字库}

本模板具体使用的中文字库是 Adobe 和 Google 合作开发的开源字体 —— 思源系列,
比 Word 里预装的中易宋体 (SimSun) 要好看一些。
此外一般高校论文模板都没有规定西文无衬线体应该用什么字体,有些直接使用黑体处理西文,
本模板采用 Arial 这一在科技论文广泛流行的西文无衬线体,
比直接用黑体要好看。

\textsf{排版知识:} 中西文字体类型应该搭配使用,不应该混用衬线和无衬线体,
正确搭配如~宋体 + Times New Roman,黑体 + Arial,
而不应该出现~黑体 + Times New Roman (常见错误)。

\subsection{字体测试}

\begin{enumerate}
	\item 中文衬线体: 思源宋体
	\item 西文衬线体: Times New Roman,1234567890
	\item 中文无衬线体: \textsf{思源黑体}
	\item 西文无衬线体: \textsf{Arial,1234567890}
\end{enumerate}

% ------------------------------------------------------------------------------
\section{页面设置}

\subsection{尺寸要求}

\begin{enumerate}
	\item A4纸张;
	\item 四边页边距均为 2.5 cm;
	\item 页眉顶部到纸张顶部距离 1.5 cm;
	\item 页脚到纸张底部距离 1.75 cm。
\end{enumerate}

\subsection{页眉页码}

\begin{enumerate}
	\item 页码格式:小五号,Time New Roman,居中;
	\item 页眉格式:五号宋体,居中;
	\item 摘要、目录、符号表的页码用罗马数字,无页眉;
	\item 正文页码用阿拉伯数字,页眉奇偶数不同;
	\item 采用带圈数字脚注 (学校未规定,其他学校有规定)。
\end{enumerate}

脚注测试~\footnote{每页脚注从1开始,下一页编号清零,重新计数。},见页脚。

% ------------------------------------------------------------------------------
\section{封面}

专业与研究方向一致,见招生简章。

\textbf{TODO}: 加个 IF 语句,如果论文题目只有一行则设置高度为 1cm, 若有两行则为 2cm。

% ------------------------------------------------------------------------------
\section{摘要}

摘要标题格式和章标题格式相同。

中文摘要关键词与正文隔一行,顶格,小四黑体,但之后的内容用小四宋体。

英文摘要关键词与正文隔一行,顶格,小四宋体加粗 (学校给的 Word 模板未统一,有些用黑体,有些直接用宋体), 
之后内容用 Times New Roman。

% ------------------------------------------------------------------------------
\section{目录}

目录格式:

\begin{enumerate}
	\item ``目录'' 标题格式同章标题格式,小二号、黑体、居中,段前 24 bp, 段后 18 bp;
	\item 目录中的章标题格式:小四号、黑体;
	\item 目录中的节标题格式:小四号、宋体。
\end{enumerate}

\textsf{排版知识:} 目录里各章节对应的页码不用统一字体,应该和其对应的章节格式相同,这样才能增加辨识度。

目录包含:

\begin{enumerate}
	\item 正文章节题目 (到三级标题,即 1.1.1);
	\item 致谢;
	\item 参考文献;
	\item 附录 (可选);
	\item 攻读学位期间发表论文 (可选);
	\item 索引 (可选)。
\end{enumerate}

\textbf{注:}目录是否包含摘要、符号表和目录本身? 学校未具体规定,其给的模板包含了。

% ------------------------------------------------------------------------------
\section{章节标题}

\subsection{格式要求}

章节标题的段前段后距离学校未规定。

\begin{enumerate}
	\item 只编号到三级标题,即1.1.1;
	\item	章标题:小二,黑体 (英文 Arial ),居中,段前 24 bp, 段后 18 bp;
	\item 一级节标题:小三,黑体 (英文 Arial ),左顶格,段前 24 bp,段后稍大于正文行距;
	\item 二级节标题:四号,黑体 (英文 Arial ),左顶格,段前 12 bp,段后稍大于正文行距;
	\item 章节序号与章节名间空一字;
	\item 论文每章应另起一页;
	\item 章标题字数一般在 15 字以内,不得使用标点符号。
\end{enumerate}

% ------------------------------------------------------------------------------
\section{数学符号}

\begin{enumerate}
	\item 物理变量、常量采用斜体,如 $y$。 物理量符号作下标时也用斜体, 如$x_y$;
	\item 计量单位采用正体。表达时刻应采用中文计量单位,
				如 ``上午8点3刻'', 不能用 ``8h54min'';
	\item 统一使用工程或国际单位,不可混用;
	\item 数字一般用阿拉伯数字。
\end{enumerate}

% ------------------------------------------------------------------------------
\section{公式}

\subsection{格式要求}

\begin{enumerate}
	\item 公式字体学校未规定,本模板使用和正文一致的 Times New Roman;
	\item 正文公式按章编号,并用 ``-'' 连接并加括号;
\end{enumerate}

\textbf{Tips:} 编辑模式下,公式环境的上方不应该加空行,否则在生成文档中会有额外空行。
但为了区分正文和公式,可以公式前加入一行无内容的注释标识,公式后可以直接空行。
但公式后不要将空行换成注释标识,那样会导致无缩进。

\subsection{字体测试}

英文斜体在公式中和正文中对比,保证都是Times New Roman。
除了字母 \textit{z} 、$z$和数字,其他都差不多,
这是因为公式字体使用的是 Times New Roman 的开源版本。

公式中: $abcdefghijklmnopqrstuvwxyz1234567890$

正文中:\textit{abcdefghijklmnopqrstuvwxyz}1234567890

\subsection{段前段后间距测试}

测试公式前长文字对公式段前距离的影响。
%
\begin{equation}  \label{eq:a}
	f(x) = \int\frac{\sin x}{x}\,\mathrm{d}x
\end{equation}

测试公式段后距离。

测试段文字:
%
\begin{equation}	\label{eq:b}
	\Omega = \sin x + \log y
\end{equation}

多个公式连写时,应该用 gather 或 align 环境, 否则会出现公式间的间距不同
(因为第一个公式前有一行文字,若该文字大于公式前缩进空白长度,会使用 displayskip,
而后面的公式因为没有一行文字在前,会使用 displayshortskip)。
%
\begin{gather}
	\lim x = \Gamma	\label{eq:c} \\
	\frac{x}{y} = z	\label{eq:d}
\end{gather}

\subsection{公式引用}

\begin{enumerate}
	\item 引用一个公式: 见\cref{eq:a};
	\item 引用两个公式: 见\cref{eq:a,eq:b};
	\item 引用多个间断公式: 见\cref{eq:a,eq:b,eq:d};
	\item 引用多个连续公式: 见\cref{eq:a,eq:b,eq:c,eq:d}。
\end{enumerate}

注意和英文写作不同,cref 引用命令前不要加\~{},否则会在 ``公式'' 前出现多余空格。

《撰写规范》要求在公式后用破折号给第一次出现的物理量解释,样式很丑,在考虑要不要用,
如果要用的话,可以定义一个环境命令调用这格式。

\noindent 式中\hspace{\ccwd} $x$ —— 物理量含义,这一行要顶格无缩进;

\hspace{\ccwd} $\Gamma$ —— 这一行要在原缩进基础上再缩进1个中文字符,保证垂向对齐;

\hspace{\ccwd} $\Gamma_\ell$ —— 
但如果改变量较宽,就无法对齐,可能需要在破折号前预留个较宽的长度。

% ------------------------------------------------------------------------------
\section{图表}

图表的允许位置建议设置为宽松的[htb]。

\subsection{表格}

表格格式要求:

\begin{enumerate}
	\item 表格不加左、右边线,即使用三线表;
	\item	表序号用如``表1-1'';
	\item 表序与表名之间空一个中文字符;
	\item 表名不允许使用标点;
	\item 表头统一单位加圆括号;
	\item 数据空缺用破折号 ``——''占位;
	\item 不允许使用 ``同上'' 之类写法。
\end{enumerate}

测试表格与表上面文字的间距。

\begin{table}[htb]
	\centering
	\caption{表格测试}
	\label{tab:a}
	\begin{tabular}{cc}
		\toprule
		表格 & 1  \\
		\midrule
		测试 & 2  \\
		\bottomrule
	\end{tabular}
\end{table}

测试表格与下面文字的间距。

表格引用,见\cref{tab:a}。

\subsection{图}

格式要求:

\begin{enumerate}
	\item	图序号用如``图1-1'';
	\item 图序与图名之间空一个中文字符;
	\item 图题与插图不能分为两页。
\end{enumerate}

测试插入模式下的图与上面文字的间距:等于行距 (\the\textfloatsep)。

\begin{figure}[h]
	\centering
	\includegraphics[scale=0.7]{example-image-b}
	\caption{图标题采用五号宋体}
	\label{fig:a}
\end{figure}

测试插入模式下的图与下面文字的间距:略大于行距。

图引用测试:

\begin{enumerate}
	\item	见\cref{fig:a};
	\item 两图引用:见\cref{fig:a,fig:b};
	\item 多图引用:见\cref{fig:a,fig:b,fig:c}。
\end{enumerate}

\clearpage

测试浮动模式下的图与文字的间距:略大于行距。

\begin{figure}[t]
	\centering
	\includegraphics[scale=0.7]{example-image-a}
	\caption{测试两浮动体间的距离:等于行距}
	\label{fig:b}
\end{figure}

\begin{figure}[t]
	\centering
	\includegraphics[scale=0.7]{example-image-a}
	\caption{图标题采用五号宋体}
	\label{fig:c}
\end{figure}

% ------------------------------------------------------------------------------
\section{列表}

使用 \textsf{enumerate} 宏包定制列表。

\begin{enumerate}
	\item 列表1
	\item 列表2
\end{enumerate}

% ------------------------------------------------------------------------------
\section{索引}


% ------------------------------------------------------------------------------
\section{参考文献}

\subsection{格式要求}

引用格式:

\begin{enumerate}
	\item 引用数字在右上角:小五号,Times New Roman;
	\item 在文中直接说明时:小四号,Times New Roman。
\end{enumerate}

文献列表格式:

\begin{enumerate}
	\item 五号,宋体 + Times New Roman;
	\item 1.25倍行距。
\end{enumerate}

\subsection{编译方式}

采用 \textbf{biblatex + biber} 方式编译参考文献。

biblatex 的优势例如: (引用刘海洋的知乎回答)

\begin{enumerate}
	\item 文献数据结构全面,可引用的内容更多。用 bibtex 一般只能引用编号,
				配合 natbib 宏包和特定的 bst 可以引用作者、年代。
				而 biblatex 还可以引用标题、URL或者其他你想引用的文献信息。
				因为文献信息以较完整的数据结构保存在生成的 bbl 文件中,所以引用的内容更多。
				这是相比 BibTeX 的一个基本设计上的不同;
	\item 输出控制由 LaTeX 代码完成,开发容易。
				bst 的语言比较晦涩,开发难度大一些,biblatex 开发就容易多了,
				所以现在第三方 biblatex 格式出现得比较多,
				用户自己调整相比改 bst 也容易一些。。
	\item 文献提取与排序使用功能更强的 biber。
				biber 支持 Unicode,支持按汉字排序,支持按文献标题排序、编辑甚至翻译者排序,等等,
				这都是 BibTeX 无法提供的。
\end{enumerate}

但biblatex 开发晚得多,所以直接提供的功能也就多很多。
当然,也因为出现得晚,期刊投稿接受的就相对少一些。

\subsection{引用测试}

一般引用的引用数字在右上角~\cite{knuth1984},
也可以与正文平排使用:见文献~\parencite{knuth1986}。

引用列表格式使用的是 《gb7714-2015》
见中文专著~\cite{liu2013}。

% 显示参考文献列表,并加参考文献加入目录
\printbibliography[heading=bibintoc]
\end{document}
